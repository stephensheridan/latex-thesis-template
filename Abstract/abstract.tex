
% Thesis Abstract -----------------------------------------------------


%\begin{abstractslong}    %uncommenting this line, gives a different abstract heading
\begin{abstracts}        %this creates the heading for the abstract page

An Abstract provides a summary of the thesis. The University Guidelines for Research Degree Programmes stipulates that an Abstract of no more than 300 words is required when submitting your thesis. A useful way to plan your Abstract is to think of it as a condensed version of the thesis in its entirety. It should provide a summary of the main sections of the thesis: the Introduction, Materials and Methods, Results and Discussion. An effective Abstract should allow readers to understand the basic content of a thesis quickly and precisely, so that they can judge whether it is relevant to their own research interests and, therefore, worthwhile reading the thesis itself. Pointing out the novelty of the work is important in this regard.

It is normal practice to present the Abstract as a single paragraph. In terms of organization, a
useful plan to follow is:

\begin{itemize}
	\item clarify the main objectives and scope of the research
	\item describe the methods employed
	\item  provide a brief summary of the results
	\item outline the key conclusions
\end{itemize}

Use the past tense when composing your Abstract – you are writing about what has been done. Do not include tables or graphs in the Abstract; also, references to literature should not be included. Make sure that your Abstract does not contain any information that is not included in your thesis. Accuracy and precision are crucial to the success of an Abstract.
 
\end{abstracts}
