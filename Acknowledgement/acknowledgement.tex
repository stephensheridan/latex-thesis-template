% Thesis Acknowledgements ------------------------------------------------


%\begin{acknowledgementslong} %uncommenting this line, gives a different acknowledgements heading
\begin{acknowledgements}      %this creates the heading for the acknowlegments


When composing the Acknowledgements section of your thesis, bear in mind that you must include two key components:

\begin{itemize}
	\item Firstly, you should acknowledge any technical assistance you received during your research - in the laboratory or elsewhere. An example of such an acknowledgement is: ‘Thanks are due to J. Smith for assistance with DNS data gathering and data analysis and to A. Smith for valuable discussion.’ Additionally, the source of any special material, cultures or equipment should be acknowledged.
	
	\item Secondly, use the Acknowledgements to refer to any external funding or financial assistance, such as fellowships, grants or contracts, which you have received during your postgraduate studies.
\end{itemize}

In general, be courteous and precise with the wording of your Acknowledgements. Say ‘Thank you’ to those who have helped you in your postgraduate research to show that you value the advice and support of your friends and colleagues.
\end{acknowledgements}
%\end{acknowledgmentslong}

% ------------------------------------------------------------------------

%%% Local Variables: 
%%% mode: latex
%%% TeX-master: "../thesis"
%%% End: 
