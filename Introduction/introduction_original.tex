%%% Thesis Introduction --------------------------------------------------
\chapter{Introduction}
\ifpdf
    \graphicspath{{Introduction/IntroductionFigs/PNG/}{Introduction/IntroductionFigs/PDF/}{Introduction/IntroductionFigs/}}
\else
    \graphicspath{{Introduction/IntroductionFigs/EPS/}{Introduction/IntroductionFigs/}}
\fi

\section{First section}

Citations can be in tow formats: \citep{Mislevy2012} or \citep{Cassidy2011}. Pages numbers are included as follows: \citep[p. 214]{Pascarella2005}. 

\section{Second section}
The following list includes labels allowing specific items to be refereces else where. The `begin' statement defines format for both how items are numbered, and how their reference will appear. This are done as a list of objectives, so the numbering is precedded by `Ob':
\begin{enumerate}[label=\emph{Ob\arabic{enumi}}.,ref=\emph{Ob\arabic{enumi}}]
%ob1 factors
\item Objective 1 is  \label{Item1}
%ob2 edm models
\item Objective 2 is  \label{Item2}
\end{enumerate}


%The data used covered a diverse student population enrolled on a range of academic courses. Both regression models predicting Grade Point Average (GPA) and classification models predicting students at risk of failing in first year of study were evaluated.




%which have good accuracy in identifying students who will pass or fail, are based on data that comes available after the student has engaged in course work.
%, by which time it can be too late to intervene. Many models also tend to be specific to a particular academic discipline. 

% Arnold2012 talks about advantage of early intervention, and Lauria2013.




\section{Next section}
Regular list

\begin{enumerate}
\item List item 1
\item List item 2
\end{enumerate}



\subsection{and a subsection} 
\begin{minipage}{0.85\textwidth}
\textbf{creating a mini page:} A minin page within the main page
\end{minipage}


%%% ----------------------------------------------------------------------


%%% Local Variables: 
%%% mode: latex
%%% TeX-master: "../thesis"
%%% End: 
