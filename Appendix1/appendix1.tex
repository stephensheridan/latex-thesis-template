\chapter{What should go into an appendix }\label{chapter:AppendixA}

\section{Introduction}
An Appendix is not an essential component of a thesis; however, it is a useful section in which to include material that is relevant to the main body of your thesis but not suitable for inclusion in it. Therefore, we can view an Appendix as a set of related items to the main body of a thesis. For example, Appendices may include:

\begin{itemize}
	\item tables that are too detailed for presentation in text
	\item large groups of illustrations
	\item technical notes on methodology
	\item forms used in collecting materials/data
	\item copies of relevant documents
	\item illustrative materials such as figures.
\end{itemize}

To organise your Appendices correctly, you must place materials of different categories in separate Appendices. When there are numerous Appendices, each is given a number or a letter: Appendix 1 or Appendix A. Also, a clear title should be provided for each section of the Appendices. Check with your supervisor which is preferable in your case. The page numbers used in the Appendices are separate from those of the main thesis.