%input macros (i.e. write your own macros file called MacroFile1.tex)
%\include{Macros/MacroFile1}

 \documentclass[oneside,11pt, hidelinks]{Classes/CUEDthesisPSnPDF}
 %If changing to onesided, uncomment line 154 (oddmargin size) i CUEDthesisPSnPDF

%Next 12 lines add the words 'Table' and 'Filgure' to the list of tables and figures GG
 \usepackage{tocloft}
\newlength{\mylen}
%code for Figures
\renewcommand{\cftfigpresnum}{\figurename\enspace}
\renewcommand{\cftfigaftersnum}{:}
\settowidth{\mylen}{\cftfigpresnum\cftfigaftersnum}
\addtolength{\cftfignumwidth}{\mylen}
%code for Tables
\renewcommand{\cfttabpresnum}{\tablename\enspace}
\renewcommand{\cfttabaftersnum}{:}
\settowidth{\mylen}{\cfttabpresnum\cfttabaftersnum}
\addtolength{\cfttabnumwidth}{\mylen}
%End of List of tables and figures formatting


%packages added by GG
\usepackage{framed}
\usepackage{mdframed}
\usepackage{multirow}
\usepackage{multicol}
\usepackage{makeidx}
 \usepackage{hyperref}
\usepackage{subcaption}
\usepackage{rotating}
\usepackage{latexsym}
 \usepackage{lscape}
 \usepackage{epstopdf}
 \usepackage{enumitem}
 \usepackage{amsmath}
  \usepackage{array}
  \usepackage{color}
\usepackage[svgnames]{xcolor}
\usepackage{arydshln}
  \usepackage{pdflscape}
  %\usepackage{siunitx}
  %\usepackage[acronym,toc]{glossaries}
  \usepackage[toc,acronym]{glossaries}
  %\makeglossary

% Create Glossaries
\newglossary{acron}{gls1}{glo1}{List of Acronyms}
\newglossary{abbrev}{gls2}{glo2}{List of Abbreviations}
\newglossary{symbol}{gls3}{glo3}{List of Symbols}
  
%\setlength{\glsdescwidth}{15cm}

%\newglossary[slg]{symbol}{syi}{syg}{List of Symbols} % create add. symbolslist
%\newglossary[syg]{symbol}{sys}{syo}{List of Symbols}
% abbreviations:
%\newglossary{abbrev}{abs}{abo}{List of Abbreviations}
%\glsaddkey{unit}{\glsentrytext{\glslabel}}{\glsentryunit}{\GLsentryunit}{\glsunit}{\Glsunit}{\GLSunit}


  \definecolor{myframecolour}{HTML}{E0E0E0}

  %Force a line break in a URL
  \expandafter\def\expandafter\UrlBreaks\expandafter{\UrlBreaks%  save the current one
  \do\a\do\b\do\c\do\d\do\e\do\f\do\g\do\h\do\i\do\j%
  \do\k\do\l\do\m\do\n\do\o\do\p\do\q\do\r\do\s\do\t%
  \do\u\do\v\do\w\do\x\do\y\do\z\do\A\do\B\do\C\do\D%
  \do\E\do\F\do\G\do\H\do\I\do\J\do\K\do\L\do\M\do\N%
  \do\O\do\P\do\Q\do\R\do\S\do\T\do\U\do\V\do\W\do\X%
  \do\Y\do\Z}


  %Define parameters for Key findings boxes GG
  \setlength{\FrameSep}{2pt}


\makeindex

\makeglossaries
\loadglsentries{phd_acronyms}
\loadglsentries{phd_abbreviations}
\loadglsentries{phd_symbols}
%\loadglsentries[symbols]{database2}



\ifpdf
    \pdfinfo { /Title  (PhD Stephen Sheridan)
               /Creator (TeX)
               /Producer (pdfTeX)
               /Author (Stephen Sheridan stephen.sheridan@itb.ie)
               /CreationDate (D:20150601000000)  %format D:YYYYMMDDhhmmss
               /ModDate (D:20150601000000)
               /Subject (subject terms)
               /Keywords (PhD, Thesis)}
    \pdfcatalog { /PageMode (/UseOutlines)
                  /OpenAction (fitbh)  }
\fi

\title{Detection of DNS-based covert channels using an Information Potential Approach}
\subtitle {\emph{A study of DNS-based covert channel communication for data exfiltration and command and control with a focus on filtering potentially damaging communications from benign DNS traffic.}}

\ifpdf
  \author{\href{mailto:stephen.sheridan@itb.ie}{Stephen Sheridan}}
    \collegeordept{Department of Informatics, \\ School of Informatics and Engineering,\\Institute of Technology Blanchardstown}

% ITB logo
%  \crest{\includegraphics[width=60mm]{itblogo}}
\else
  \author{Stephen Sheridan}
    \collegeordept{{Supervised by:}}
% insert below the file name that contains the crest in-place of 'UnivShield'
%  \crest{\includegraphics[bb = 0 0 292 336, width=30mm]{itblogo}}
\fi
%
% insert below the file name that contains the crest in-place of 'UnivShield'
% \crest{\IncludeGraphicsW{UnivShield}{40mm}{14 14 73 81}}
%

\renewcommand{\submittedtext}{A thesis submitted to Institute of Technology Blanchardstown in fulfilment of the requirements for the degree of}
\degree{Doctor of Philosophy}


\ifpdf
  \supervisorone{Principal supervisors:\\ Dr. Anthony Keane \& Dr. Brian Nolan}      %Change to one supervisor if needed
  \university{\href{http://www.itb.ie}{Institute of Technology Blanchardstown, Dublin, Ireland}}
% ITB logo
%  \crest{\includegraphics[width=60mm]{itblogo}}
\else
  \collegeordept{Department of Informatics}
  \university{Institute of Technology Blanchardstown, Dublin, Ireland}
% insert below the file name that contains the crest in-place of 'UnivShield'
%  \crest{\includegraphics[bb = 0 0 292 336, width=30mm]{itblogo}}
\fi


\degreedate{July 16th, 2017}

% turn of those nasty overfull and underfull hboxes
\hbadness=10000
\hfuzz=50pt

% Put all the style files you want in the directory StyleFiles and usepackage like this:
\usepackage{StyleFiles/watermark}

% Comment out the next line to get single spacing
\onehalfspacing

\begin{document}
Remove this page

% This seems to add a blank page to the beginning of the document
\glsaddall

%\language{english}
% A page with the abstract on including title and author etc may be
% required to be handed in separately. If this is not so, then comment
% the below 3 lines (between '\begin{abstractseparte}' and
% 'end{abstractseparate}'), normally like a declaration ... needs some more
% work, mind as environment abstracts creates a new page!
% \begin{abstractseparate}
%   
% Thesis Abstract -----------------------------------------------------


%\begin{abstractslong}    %uncommenting this line, gives a different abstract heading
\begin{abstracts}        %this creates the heading for the abstract page

An Abstract provides a summary of the thesis. The University Guidelines for Research Degree Programmes stipulates that an Abstract of no more than 300 words is required when submitting your thesis. A useful way to plan your Abstract is to think of it as a condensed version of the thesis in its entirety. It should provide a summary of the main sections of the thesis: the Introduction, Materials and Methods, Results and Discussion. An effective Abstract should allow readers to understand the basic content of a thesis quickly and precisely, so that they can judge whether it is relevant to their own research interests and, therefore, worthwhile reading the thesis itself. Pointing out the novelty of the work is important in this regard.

It is normal practice to present the Abstract as a single paragraph. In terms of organization, a
useful plan to follow is:

\begin{itemize}
	\item clarify the main objectives and scope of the research
	\item describe the methods employed
	\item  provide a brief summary of the results
	\item outline the key conclusions
\end{itemize}

Use the past tense when composing your Abstract – you are writing about what has been done. Do not include tables or graphs in the Abstract; also, references to literature should not be included. Make sure that your Abstract does not contain any information that is not included in your thesis. Accuracy and precision are crucial to the success of an Abstract.
 
\end{abstracts}

% \end{abstractseparate}

% Using the watermark package which is in StyleFiles/
% and to remove DRAFT COPY ONLY appearing on the top of all pages comment out below line
\watermark{DRAFT COPY ONLY}

\maketitle
%\makeglossaries


%set the number of sectioning levels that get number and appear in the contents
\setcounter{secnumdepth}{3}
\setcounter{tocdepth}{3}

\frontmatter % book mode only
\pagenumbering{roman}
% Thesis Dedictation ---------------------------------------------------

\begin{dedication} %this creates the heading for the dedication page

This thesis is dedicated to .........

\end{dedication}

% ----------------------------------------------------------------------

%%% Local Variables: 
%%% mode: latex
%%% TeX-master: "../thesis"
%%% End: 

% Thesis Acknowledgements ------------------------------------------------


%\begin{acknowledgementslong} %uncommenting this line, gives a different acknowledgements heading
\begin{acknowledgements}      %this creates the heading for the acknowlegments


When composing the Acknowledgements section of your thesis, bear in mind that you must include two key components:

\begin{itemize}
	\item Firstly, you should acknowledge any technical assistance you received during your research - in the laboratory or elsewhere. An example of such an acknowledgement is: ‘Thanks are due to J. Smith for assistance with DNS data gathering and data analysis and to A. Smith for valuable discussion.’ Additionally, the source of any special material, cultures or equipment should be acknowledged.
	
	\item Secondly, use the Acknowledgements to refer to any external funding or financial assistance, such as fellowships, grants or contracts, which you have received during your postgraduate studies.
\end{itemize}

In general, be courteous and precise with the wording of your Acknowledgements. Say ‘Thank you’ to those who have helped you in your postgraduate research to show that you value the advice and support of your friends and colleagues.
\end{acknowledgements}
%\end{acknowledgmentslong}

% ------------------------------------------------------------------------

%%% Local Variables: 
%%% mode: latex
%%% TeX-master: "../thesis"
%%% End: 


% Thesis Abstract -----------------------------------------------------


%\begin{abstractslong}    %uncommenting this line, gives a different abstract heading
\begin{abstracts}        %this creates the heading for the abstract page

An Abstract provides a summary of the thesis. The University Guidelines for Research Degree Programmes stipulates that an Abstract of no more than 300 words is required when submitting your thesis. A useful way to plan your Abstract is to think of it as a condensed version of the thesis in its entirety. It should provide a summary of the main sections of the thesis: the Introduction, Materials and Methods, Results and Discussion. An effective Abstract should allow readers to understand the basic content of a thesis quickly and precisely, so that they can judge whether it is relevant to their own research interests and, therefore, worthwhile reading the thesis itself. Pointing out the novelty of the work is important in this regard.

It is normal practice to present the Abstract as a single paragraph. In terms of organization, a
useful plan to follow is:

\begin{itemize}
	\item clarify the main objectives and scope of the research
	\item describe the methods employed
	\item  provide a brief summary of the results
	\item outline the key conclusions
\end{itemize}

Use the past tense when composing your Abstract – you are writing about what has been done. Do not include tables or graphs in the Abstract; also, references to literature should not be included. Make sure that your Abstract does not contain any information that is not included in your thesis. Accuracy and precision are crucial to the success of an Abstract.
 
\end{abstracts}




\tableofcontents
\newpage
\listoftables
\newpage
\listoffigures
%\include{AcronymsPhD}

%\include{phd-acronyms}

\printglossary[type=acron, nonumberlist]
\printglossary[type=abbrev, nonumberlist]
\printglossary[type=symbol, nonumberlist]

%\glsnogroupskipfalse
%\printglossary[title={List of Acronyms},type=acronym,nonumberlist]
%\printglossary[type=abbrev,nonumberlist]
% the `index' style displays the symbol, an alternative might be the `tree'
% type; one could also define a custom style
%\printglossary[type=symbol,nonumberlist]

%\include{acronymsPhD}
%\printglossary

%\printglossary[type=\acronymtype]
%\printglossaries
%\printglossary[type=\acronymtype] %,style=long]  % list of acronyms
%\printglossary[type=symbol,style=symbunitlong]   % list of symbols
%\printglossary[type=main]                     % main glossary
%\printnoidxglossaries
%\printnomenclature  %% Print the nomenclature
%\addcontentsline{toc}{chapter}{Acronyms}

\mainmatter % book mode only
%%% Thesis Introduction --------------------------------------------------
\chapter{Introduction}\label{chapter:introduction}

\section{Using citations}\label{sec:citations}
This section will outline the use of the \textbf{cite} and \textbf{citep} commands. Before can use the citation commands you will need to compile a bibliography file using one of a number of tools such as RefWorks\footnote{\url{https://www.refworks.com}}, PaperPile\footnote{\url{https://www.paperpile.com}} or Mendeley\footnote{\url{https://www.mendeley.com}}. Once you have compiled your bibliography file and exported it as a \textbf{.bib} file, you should include this file in your \LaTeX{} document by including the following line in your \textbf{thesis.tex} file. \\

\begin{lstlisting}[caption={Including your bibliography file}, numbers=none, label={lst:bibliography}]
\bibliography{References/Bibliography}
\end{lstlisting}

Using the code in Listing \ref{lst:bibliography} will ensure that \LaTeX{} will look for a file called \textbf{Bibliography.bib} in a sub folder called \textbf{References}. You will find this command included towards the end of the \textbf{thesis.tex} source file. \\

\textbf{Note:} the footnote superscripts for RefWorks, PaperPile and Mendelay were created with the following \LaTeX{} code:

\begin{lstlisting}[caption={Using the \textbf{footnote} command}, numbers=none, label={lst:footnote}]
RefWorks\footnote{\url{https://www.refworks.com}},
PaperPile\footnote{\url{https://www.paperpile.com}} or
Mendeley\footnote{\url{https://www.mendeley.com}}
\end{lstlisting}

\subsection{Cite command}\label{sec:cite-command}
The \textbf{cite} command inserts the author(s) name and parenthesis's the year of publication. The \textbf{cite} command takes the key-value of .bib file entry for the particular citation that you want to include. The following \LaTeX{} code demonstrates the use of the \textbf{cite} command. \\

\begin{lstlisting}[caption={Using the \textbf{cite} command}, numbers=none, label={lst:cite}]
\cite{Lampson1973-kx} can be attributed with introducing the term...
\end{lstlisting}

The \LaTeX{} code in Listing \ref{lst:cite} produces the following output:

\begin{mdframed}
\cite{Lampson1973-kx} can be attributed with introducing the term "covert channel" in his work entitled "Note on the confinement problem".
\end{mdframed}

\subsection{Citep command}\label{sec:citep-command}
The \textbf{citep} command parenthesis's both the author(s) name and year of publication. The \textbf{citep} command takes the key-value of .bib file entry for the particular citation that you want to include. The following \LaTeX{} code demonstrates the usage of the \textbf{citep} command. \\

\begin{lstlisting}[caption={Using the \textbf{citep} command}, numbers=none, label={lst:citep}]
The term covert channel \citep{Lampson1973-kx} is used to describe...
\end{lstlisting}

The \LaTeX{} code ion Listing \ref{lst:citep} produces the following output:

\begin{mdframed}
The term covert channel \citep{Lampson1973-kx} is used to describe the embedding of a hidden message in a carrier channel that will not raise suspicion.
\end{mdframed}


\subsection{Citing multiple authors}\label{sec:cite-multiple}
If you need to include multiple publications in the same citation, you can specify multiple citation keys in a commented list using the \textbf{cite} or \textbf{citep} command. The following \LaTeX{} snippet shows how to do this.

\begin{lstlisting}[caption={Citing multiple authors}, numbers=none, label={lst:cite-multiple}]
...DNS can be used as a covert channel \citep{Born2010-do,Aiello2012-sg}.
\end{lstlisting}

Using the \LaTeX{} code in Listing \ref{lst:cite-multiple} will produce the following output:

\begin{mdframed}
It is widely accepted that the DNS can be used as a covert channel \citep{Born2010-do,Aiello2012-sg}.
\end{mdframed}

\section{Including graphical figures}\label{sec:first-section}
We can include graphics using the following \LaTeX code. Note the size of the image can be varied by changing $0.5$ proportion in the width clause.

\begin{lstlisting}[caption={Including a graphical figure}, numbers=none, label={lst:figure}]
\begin{figure}[!ht]
\centering
\includegraphics[width=0.5\textwidth]{./images/DGR_4QDUIAAEnsE.jpg}
\caption{Using 0.6 divided by textwidth.}
\label{fig:writing-your-thesis}
\end{figure}
\end{lstlisting}

Using the \LaTeX{} code in Listing \ref{lst:figure} will produce the following output:

\begin{figure}[!ht]
	\centering
	\includegraphics[width=0.5\textwidth]{./Figures/DGR_4QDUIAAEnsE.jpg}
	\caption{Using 0.5 divided by textwidth.}
	\label{fig:writing-your-thesis}
\end{figure}

Changing the $0.5$ proportion in the \LaTeX{} code above to $0.3$ results in the following:

\begin{figure}[!ht]
	\centering
	\includegraphics[width=0.3\textwidth]{./Figures/DGR_4QDUIAAEnsE.jpg}
	\caption{Using 0.3 divided by textwidth.}
	\label{fig:writing-your-thesis-smaller}
\end{figure}

\pagebreak

\section{Using tables}\label{sec:second-section}
We can include tabular data using the following \LaTeX{} code:
\begin{lstlisting}[caption={Formatting tabular data}, numbers=none, label={lst:table}]
\begin{table}[!ht]
\begin{center}
\caption{Write the caption for your table here.}
\label{table:first-table}
\begin{tabular}{p{2cm} p{2cm} p{2cm} p{2cm} p{2cm}}
\hline
\\[-1em]
& A& B& C& D\\
\hline
Mean&   22&      22&     23&    28\\
Min&      2&        2&       4&      2\\
Max&     250&    248&    251&  251\\
\hline
\end{tabular}
\end{center}
\end{table}
\end{lstlisting}

The \LaTeX{} code in Listing \ref{lst:table}  produces the following output:

\begin{table}[!ht]
	\begin{center}
		\caption{Write the caption for your table here.}
		\label{table:first-table}
		\begin{tabular}{p{2cm} p{2cm} p{2cm} p{2cm} p{2cm}}
			\hline
			\\[-1em]
			& A& B& C& D\\
			\hline
			Mean&   22&      22&     23&    28\\
			Min&      2&        2&       4&      2\\
			Max&     250&    248&    251&  251\\
			\hline
		\end{tabular}
	\end{center}
\end{table}

\textbf{NOTE:} Presenting tabular data in \LaTeX{} is a bit of an art form. It can be very frustrating at times but stick with it as the end result always looks better for it.

\pagebreak

\section{Using lists}\label{sec:using-lists}
This section will outline how to use enumerations with various different styles.

\subsection{List of objectives}\label{sec:objective-list}
We can use enumerated lists for our objectives. Note the use of labels so we can refer to specific objectives from anywhere in the document.

\begin{lstlisting}[caption={Objectives enumeration}, numbers=none, label={lst:objectives}]
\begin{enumerate}[label=\emph{Ob\arabic{enumi}}.,ref=\emph{Ob\arabic{enumi}}]
\item Objective. \label{Ob:1}
\item Objective. \label{Ob:2}
\item Objective. \label{Ob:3} 
\end{enumerate}
\end{lstlisting}

The \LaTeX code in Listing \ref{lst:objectives} produces the following output:

\begin{mdframed}
\begin{enumerate}[label=\emph{Ob\arabic{enumi}}.,ref=\emph{Ob\arabic{enumi}}]
	\item Objective. \label{Ob:1}
	\item Objective. \label{Ob:2}
	\item Objective. \label{Ob:3} 
\end{enumerate}
\end{mdframed}

\subsection{List of research questions}\label{sec:second-subsection}
We can use enumerated lists for our research questions. \textbf{Note}: the use of labels so we can refer to specific research questions from anywhere in the document.

\begin{lstlisting}[caption={Question enumeration}, numbers=none, label={lst:questions}]
\begin{enumerate}[label=\emph{Q\arabic{enumi}}.,ref=\emph{Q\arabic{enumi}}]
\item Question?\label{Q:1} \\
\textit{This thesis will also address the following secondary research questions:}
\item Question?  \label{Q:2}
\item Question?  \label{Q:3}
\end{enumerate}
\end{lstlisting}

The \LaTeX code in Listing \ref{lst:questions} produces the following output:

\begin{mdframed}
\begin{enumerate}[label=\emph{Q\arabic{enumi}}.,ref=\emph{Q\arabic{enumi}}]
	\item Question?\label{Q:1} \\
	\textit{This thesis will also address the following secondary research questions:}
	\item Question?  \label{Q:2}
	\item Question?  \label{Q:3}
\end{enumerate}
\end{mdframed}

\subsection{Using a numbered list}\label{sec:numbered-list}
We can use a straight forward numbered list as follows:

\begin{lstlisting}[caption={Question enumeration}, numbers=none, label={lst:numbered}]
\begin{enumerate}
\item First Item
\item Second Item
\item Third Item
\end{enumerate}
\end{lstlisting}

The \LaTeX{} code in Listing \ref{lst:numbered} produces the following output:

\begin{mdframed}
\begin{enumerate}
\item First Item
\item Second Item
\item Third Item
\end{enumerate}
\end{mdframed}

\subsection{Using a bullet list}\label{sec:bullet-list}
We can use a straight forward bullet list as follows:

\begin{lstlisting}[caption={Itemised listing}, numbers=none, label={lst:bullet}]
\begin{itemize}
\item First Item
\item Second Item
\item Third Item
\end{itemize}
\end{lstlisting}

\pagebreak

The \LaTeX{} code in Listing \ref{lst:bullet} produces the following output: \\

\begin{mdframed}
\begin{itemize}
	\item First Item
	\item Second Item
	\item Third Item
\end{itemize}
\end{mdframed}


\chapter{Literature Review}\label{chapter:lit-review}

\section{Introduction}\label{sec:lit-review-intro}
A well-written, comprehensive literature review should provide examiners with evidence that a research student has made the required effort to master his or her field of knowledge. A successful literature review should have two essential features: firstly, it should evaluate the relevant literature, rather than merely cite it; secondly, it should relate the material under review to the actual thesis itself.

\section{First section}\label{sec:lit-review-first-section}
First section here.

\section{Summary}\label{sec:lit-review-summary}
It's only manners to have a summary section. Remind the reader what you have covered. This section should also highlight key findings and gaps in the research that support your project objectives and research questions.
%!TEX root = ../thesis.tex

\chapter{Methodology}\label{chapter:methodology}

\section{Introduction}\label{sec:methodology-intro}
Always include a section introduction. What is the purpose of this chapter? 

\section{First section}\label{sec:methodology-first-section}
First section here.

\section{Summary}\label{sec:methodology-summary}
It's only manners to have a summary section. Remind the reader what you have covered. 

%!TEX root = ../thesis.tex
\chapter{Results} \label{chapter:results}

\section{Introduction}\label{sec:results-intro}
Always include a section introduction. What is the purpose of this chapter?  NOTE: this chapter could be split in two. A separate chapter for Results and one for Analysis. This will really depend on this structure of your thesis. Consult with your project supervisor.

\section{First section}\label{sec:results-first-section}
First section here.

\section{Summary}\label{sec:results-summary}
It's only manners to have a summary section. Remind the reader what you have covered. 
\include{Analysis/analysis}
\chapter{Conclusion}\label{chapter:conclusion}

\section{Introduction}\label{sec:conclusion-intro}
Always include a section introduction. What is the purpose of this chapter?  

\section{First section}\label{sec:conclusion-first-section}
First section here.

\section{Future work}\label{sec:conclusion-future}
Outline future work.

\section{Conclusion}\label{sec:conclusion-conclusion}
It's only manners to have a summary section. Remind the reader what you have covered. 




%\backmatter % book mode only, turns off chapter numbering

\bibliographystyle{chicago}
\renewcommand{\bibname}{References} % changes default name Bibliography to References
\renewcommand{\bibfont}{\small}
\begin{singlespace}
%\setlength\bibitemsep{10pt}
\setlength{\itemsep}{0pt}

\bibliography{References/PhDBibliography}
\end{singlespace}

\appendix
\chapter{What should go into an appendix }\label{chapter:AppendixA}

\section{Introduction}
An Appendix is not an essential component of a thesis; however, it is a useful section in which to include material that is relevant to the main body of your thesis but not suitable for inclusion in it. Therefore, we can view an Appendix as a set of related items to the main body of a thesis. For example, Appendices may include:

\begin{itemize}
	\item tables that are too detailed for presentation in text
	\item large groups of illustrations
	\item technical notes on methodology
	\item forms used in collecting materials/data
	\item copies of relevant documents
	\item illustrative materials such as figures.
\end{itemize}

To organise your Appendices correctly, you must place materials of different categories in separate Appendices. When there are numerous Appendices, each is given a number or a letter: Appendix 1 or Appendix A. Also, a clear title should be provided for each section of the Appendices. Check with your supervisor which is preferable in your case. The page numbers used in the Appendices are separate from those of the main thesis.
\chapter{Abstracts from Publications Emanating from this Study }\label{chapter:AppendixB}

\section{Peer-reviewed journals}

\section{Conference papers}

\section{Talks and presentations}
\chapter{Code listings}
\label{AppendixC}

\section{Java main()}
\begin{lstlisting}[caption={Java main()}]
public static void main(String args[]){
	System.out.println("Hello world!");
}
\end{lstlisting}

\section{Python function}
\begin{lstlisting}[caption={Python function}]
def addition(val1, val2):
	return val1 + val2
\end{lstlisting}

\include{Appendix4/appendix4}



\end{document}
